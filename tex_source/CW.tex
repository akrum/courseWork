\documentclass[12pt]{article}
\usepackage[utf8]{inputenc}
\usepackage[russian]{babel}
\usepackage{amssymb,amsmath}
\usepackage{amsmath}
\usepackage{amsthm}
\usepackage{graphicx}
\newtheorem{theorem}{Теорема}
\newenvironment{rowequmat}[1]{\left(\array{@{}#1@{}}}{\endarray\right)}
\textheight=24cm
\textwidth=16cm
\oddsidemargin=0pt
\topmargin=-1.5cm
\parindent=24pt
\title{Курсовой проект}
\author{\copyright Андрей Румянцев}
\date{29 ноября 2016}
\begin{document}
\begin{titlepage}
    \linespread{1.1}
    \begin{center}
    \fontsize{15pt}{15pt}\selectfont
    МИНЕСТЕРСТВО ОБРАЗОВАНИЯ РЕСПУБЛИКИ БЕЛАРУСЬ\\
    \vspace{0.5cm}
    БЕЛОРУССКИЙ ГОСУДАРСТВЕННЫЙ УНИВЕРСИТЕТ\\
    \vspace{0.5cm}
    \textit{ФАКУЛЬТЕТ ПРИКЛАДНОЙ МАТЕМАТИКИ И ИНФОРМАТИКИ}\\
    \vspace{0.5cm}
    \textit{КАФЕДРА МАТЕМАТИЧЕСКОГО МОДЕЛИРОВАНИЯ И АНАЛИЗА ДАННЫХ}\\
    \vspace{3.5cm}
    \fontsize{18pt}{18pt}\selectfont
    Румянцев\\
    Андрей Кириллович\\
    \vspace{0.5cm}
    \textbf{"Робастные оценки параметров регрессии при наличии группированой выборки"}\\
    \vspace{0.5cm}
    \fontsize{16pt}{16pt}\selectfont
    Курсовой проект\\
    \end{center}
    \vspace{3.5cm}
    \fontsize{14pt}{14pt}\selectfont
    \hspace{-0.25cm}
    \def\arraystretch{1.2}
    \begin{tabular}{l@{\hspace{3.25cm}}l}
    Допущен к защите & Научный руководитель:\\
    <<\underline{~~~~}>>~~\underline{~~~~~~~~~~~~} 2017 г&Агеева Елена Сергеевна\\
    
    Ассистент кафедры математического \\
    моделирования и анализа данных ФПМИ,  \\
    кандидат физико-математических наук,\\
    Агеева Елена Сергеевна
    
    \end{tabular}
    \vspace{3cm}
    \begin{center}
    \fontsize{16pt}{16pt}\selectfont
    Минск, 2017
    \end{center}
  \end{titlepage}
\newpage
\tableofcontents
\newpage
\section{Введение}
Существует несколько подходов для оценки параметров регрессии, но далеко не все устойчивы к возникновениям аномальных наблюдений.
В реальной жизни аномальные наблюдения возникают постоянно, поэтому большинство методов просто неприменимо.
В прошлом веке в работах Хьюбера была заложена теория робастного оценивания.\hfill\break
Были предложены следующие робастные оценки\cite{Huber}:\hfill\break
\begin{itemize}
    \item М-Оценки\\
    \item R-Оценки\\
    \item L-Оценки
\end{itemize}
М-оценки -- некоторое подобие оценок максимального правдоподобия(ММП-оценки - частный случай), L-оценки строятся на основе линейных комбинаций порядковых статистик, R-оценки -- на основе ранговых статистик.
В данном курсовом проекте я буду моделировать функцию регрессии с аномальными наблюдениями, анализировать точность методов и находить для разных методов так называемый ''breakpoint''--процент аномальных наблюдений, при котором увеличение количества наблюдений не повысит точность методов.


\section{Теоретические сведения}
На данном этапе будем работать с линейной регрессией:\hfill\break
\begin{eqnarray}
    y_i=\alpha+\beta_1 x_{i1}+\beta_2 x_{i2}+\dots+\beta_n x_{in}+\epsilon_i,
\end{eqnarray}
Где $y_i$ -- i-е наблюдение из N наблюдений, $x_i$ регрессоры, \{$\alpha,\beta_k, k=\overline{1,n}$\}-- параметры регрессии, а $\epsilon_i$ -- ошибка, распределение которой подчиняется нормальному закону с нулевым ожиданием и дисперсией $\sigma^2$.\hfill\break
В нашей задаче считаем параметры \{$\alpha,\beta_k, k=\overline{1,n}$\} неизвестными, их нам и требуется найти.\hfill\break
Теперь рассмотрим некоторые методы оценки параметров регрессии:
\subsection{Метод Наименьших Квадратов}
\subsection{М-оценки}
\subsection{L-оценки}


\section{Моделирование регрессии на языке Python}
Подключим необходимые библиотеки:\hfill\break
\begin{verbatim}
import numpy as np
import matplotlib.pyplot as plt
from random import random
import pylab
import scipy
from outliers import smirnov_grubbs as grubbs
from matplotlib.backends.backend_pdf import PdfPages
from statsmodels.robust.scale import mad
import theano
import theano.tensor as T
import statsmodels.api as sm
import statsmodels.formula.api as smf
\end{verbatim}
Заведем константы для моделирования: количество наблюдейний, процент аномальных наблюдений, и параметры регрессии, использующиеся в моделировании:
\begin{verbatim}
SAMPLE_QUINTITY=100
OUTLIER_PERCENTAGE = 10.0
regressionParameters = np.matrix([100,4]).T
\end{verbatim}
Проинициализируем результирующий вектор y:
\begin{verbatim}
y_points = np.zeros(shape = SAMPLE_QUINTITY)
\end{verbatim}
Теперь моделируем y:
\begin{verbatim}
x_points = np.zeros(shape=[SAMPLE_QUINTITY,len(regressionParameters)])
y_points = np.zeros(shape = SAMPLE_QUINTITY)
# plt.plot(x_points,y_points,'ro')
# # plt.hist(y_points,bins="auto")
# plt.show()
for i in range(0,SAMPLE_QUINTITY):
    if random()>OUTLIER_PERCENTAGE/100:
        x_points[i] = np.append(np.ones(1),np.random.uniform(-5,5,size = len(regressionParameters)-1))
        # print(x_points[i])
        y_points[i]=(x_points[i]*regressionParameters)+np.random.normal(0,4)
    else:
        x_points[i] = np.append(np.ones(1),np.random.uniform(-5,5,size = len(regressionParameters)-1))
        y_points[i]=np.random.normal(100,10, size=1)
plt.plot(x_points.T[1],y_points,'ro')
plt.show()
\end{verbatim}
Программа выводит такой график:
\newpage
\begin{thebibliography}{9}
    \bibitem{Huber}
    Хьюбер Дж П.,
    \textit{Робастность в статистике:пер. с англ.}.
    М.:Мир,1984-304с

    \bibitem{Knuth}

\end{thebibliography}
\end{document}