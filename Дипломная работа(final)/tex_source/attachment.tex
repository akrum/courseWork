\section*{Приложение}
\phantomsection
\addcontentsline{toc}{section}{Приложение}

Метод наименьших квадратов по центрам интервалов
\begin{Verbatim}[fontsize=\scriptsize]
    class ApproximationGEMModelNaive(ApproximationGEMModelRedesigned):
        def fit(self):
            self.classify()
    
            def ex_generator(mu_data):
                for i in range(0, self.endogen.size):
                    if mu_data[i] is None:
                        continue
                    a_mu_i_plus_1 = mu_data[i] * Defines.INTERVAL_LENGTH
                    a_mu_i = mu_data[i] * Defines.INTERVAL_LENGTH - Defines.INTERVAL_LENGTH
                    yield (a_mu_i_plus_1 + a_mu_i) / 2
    
            naive_ex_data_positive = np.fromiter(ex_generator(self._np_freq_positive), float)
            naive_ex_data_negative = np.fromiter(ex_generator(self._np_freq_negative), float)
    
            naive_ex_data_full = np.append(naive_ex_data_positive, naive_ex_data_negative)
    
            z, resid, rank, sigma = np.linalg.lstsq(self.exogen, naive_ex_data_full, rcond=None)
            return z
\end{Verbatim}

Моделирование полиномиальной регрессии:
\begin{Verbatim}[fontsize=\scriptsize]
def modulate_polynomial_regression(regression_sample_quintity, regression_outlier_percentage):
    regression_parameters = ACCURATE_RESULT
    _x_points = np.zeros(shape=[regression_sample_quintity, len(regression_parameters)])
    _y_points = np.zeros(shape=regression_sample_quintity)

    def np_random_polynomial(size):
        _res = np.zeros(size)
        for i in range(0, size):
            _res[i] = random.uniform(-5, 5) ** (i + 1)

        return _res

    for i in range(0, regression_sample_quintity):
        _x_points[i] = np.append(np.ones(1), np_random_polynomial(len(ACCURATE_RESULT) - 1))
        if random.random() > regression_outlier_percentage / 100:
            _y_points[i] = (_x_points[i] * ACCURATE_RESULT) + np.random.normal(0, 4)
        else:
            _y_points[i] = np.random.normal(100.0, 15.0, size=1)

    return _x_points, _y_points
\end{Verbatim}

Моделирование линейной регрессии:
\begin{Verbatim}[fontsize=\scriptsize]
    def modulateRegression(regression_sample_quintity, regression_outlier_percentage):
    regression_parameters = ACCURATE_RESULT
    _x_points = np.zeros(shape=[regression_sample_quintity, len(regression_parameters)])
    _y_points = np.zeros(shape=regression_sample_quintity)

    for i in range(0, regression_sample_quintity):
        if random.random() > regression_outlier_percentage / 100:
            _x_points[i] = np.append(np.ones(1), np.random.uniform(-5, 5, size=len(regression_parameters) - 1))
            _y_points[i] = (_x_points[i] * regression_parameters) + np.random.normal(0, 4)
        else:
            _x_points[i] = np.append(np.ones(1), np.random.uniform(-5, 5, size=len(regression_parameters) - 1))
            _y_points[i] = np.random.normal(100.0, 15.0, size=1)

    return _x_points, _y_points
\end{Verbatim}

\textit{Метод наименьших квадратов по центрам интервалов}:
\begin{Verbatim}[fontsize=\scriptsize]

def fit_data_naive_classic():
    sample_sizes = []
    all_results_classic = []
    all_results_naive = []
    for sample_size in range(SAMPLE_SIZE_MIN, SAMPLE_SIZE_MAX+1, SAMPLE_SIZE_STEP):
        successful_fit = False
        while not successful_fit:
            x_points, y_points = modulateRegression(sample_size, OUTLIER_PERCENTAGE)
            approx_model = groupingEstimates.GEM(x_points, y_points)
            approx_model_naive = groupingEstimatesNaive.GEM_N(x_points, y_points)
            try:
                result = approx_model.fit()
                print("GEM {}".format(result))
                result_naive = approx_model_naive.fit()
                print("GEM_N {}".format(result_naive))

                successful_fit = True

                all_results_classic.append(result)
                all_results_naive.append(result_naive)
                sample_sizes.append(sample_size)
            except KeyboardInterrupt:
                print("stopping...")
                np.save(NP_DATA_PATH + "gem_res_classic", all_results_classic)
                np.save(NP_DATA_PATH + "gem_res_naive", all_results_naive)
                np.save(NP_DATA_PATH + "gem_sizes", sample_sizes)
                quit()
            except Exception as e:
                print(e)
    np.save(NP_DATA_PATH + "gem_res_classic", all_results_classic)
    np.save(NP_DATA_PATH + "gem_res_naive", all_results_naive)
    np.save(NP_DATA_PATH + "gem_sizes", sample_sizes)

\end{Verbatim}

График с разным объемом выборки:
\begin{Verbatim}[fontsize=\scriptsize]
def plot_with_different_sample_size():
    sample_sizes = []
    all_results_with_classification = []
    all_results_without_classification = []

    x_points = None
    y_points = None

    for sample_size in range(SAMPLE_SIZE_MIN, SAMPLE_SIZE_MAX+1, SAMPLE_SIZE_STEP):
        successful_fit = False
        while not successful_fit:
            x_points_t, y_points_t = modulateRegression(sample_size, OUTLIER_PERCENTAGE)

            if x_points is None or y_points is None:
                x_points = x_points_t
                y_points = y_points_t
            else:
                x_points = np.append(x_points, x_points_t, axis=0)
                y_points = np.append(y_points, y_points_t, axis=0)

            approx_model = groupingEstimates.GEM(x_points, y_points)
            try:
                result = approx_model.fit()
                print("GEM {}".format(result))
                result_without = approx_model.fit_without_reclassification()
                print("GEM_without {}".format(result_without))

                successful_fit = True

                all_results_with_classification.append(result)
                all_results_without_classification.append(result_without)
                sample_sizes.append(sample_size)
            except KeyboardInterrupt:
                print("stopping...")
                np.save(NP_DATA_PATH + "gem_res_with", all_results_with_classification)
                np.save(NP_DATA_PATH + "gem_res_without", all_results_without_classification)
                np.save(NP_DATA_PATH + "gem_sizes_with_without", sample_sizes)
                quit()
            except Exception as e:
                print(e)
    np.save(NP_DATA_PATH + "gem_res_with", all_results_with_classification)
    np.save(NP_DATA_PATH + "gem_res_without", all_results_without_classification)
    np.save(NP_DATA_PATH + "gem_sizes_with_without", sample_sizes)

\end{Verbatim}

График с разным уровнем переклассификации:
\begin{Verbatim}[fontsize=\scriptsize]
def plot_with_different_reclassification_level():
    reclassification_levels = []
    all_results_with_classification = []
    recl_level_min = 10
    recl_level_max = 40

    x_points, y_points = modulateRegression(500, OUTLIER_PERCENTAGE)

    for recl_level in range(recl_level_min, recl_level_max + 1, 2):
        GroupingEstimatesDefines.RECLASSIFICATION_LEVEL = recl_level

        successful_fit = False
        while not successful_fit:
            approx_model = groupingEstimates.GEM(x_points, y_points)
            try:
                result = approx_model.fit()
                print("GEM {}".format(result))

                successful_fit = True

                all_results_with_classification.append(result)
                reclassification_levels.append(recl_level)
            except KeyboardInterrupt:
                print("stopping...")
                np.save(NP_DATA_PATH + "gem_with_dif_level_results", all_results_with_classification)
                np.save(NP_DATA_PATH + "gem_with_dif_level_levels", reclassification_levels)
                quit()
            except Exception as e:
                print(e)
    np.save(NP_DATA_PATH + "gem_with_dif_level_results", all_results_with_classification)
    np.save(NP_DATA_PATH + "gem_with_dif_level_levels", reclassification_levels)

\end{Verbatim}