\begin{center}
    \section*{ВВЕДЕНИЕ}
\end{center}
\phantomsection
\addcontentsline{toc}{section}{ВВЕДЕНИЕ}

В математической статистике широко используется регрессионная модель. 
Одним из примеров использования такой модели является автономный мониторинг целостности приемника (RAIM) системы глобального позиционирования (GPS). 
Это технология, которая разработана для оценки целостности сигналов GPS. 
Такая система очень важна в приложениях авиационной и морской навигации, где безопасность приложений GPS критична \cite{GPS_POSITIONING}.
Существует несколько подходов для оценки параметров регрессии, но далеко не все устойчивы к возникновениям искажений, 
то есть таких наблюдений, которые не подчиняются общей модели. 
На практике же аномальные наблюдения возникают постоянно. 
Такие наблюдения могут возникать по разным причинам: из-за ошибки измерения, из-за необычной природы входных данных и др \cite{OLSforGrouping}.
По этой причине большинство классических методов неприменимо.
В прошлом веке в работах Хьюбера была заложена теория робастного оценивания, т.~е. такого оценивания, которое исключает влияние искажений в выборке на результат оценивания.

Были предложены следующие робастные оценки\cite{Huber}:
\begin{itemize}
    \item М-Оценки
    \item R-Оценки
    \item L-Оценки
\end{itemize}
М-оценки -- некоторое подобие оценок максимального правдоподобия (ММП-оценки - частный случай), L-оценки строятся на основе линейных комбинаций порядковых статистик, R-оценки -- на основе ранговых статистик.

Такие случаи, когда зависимые переменные наблюдаются с выбросами или с пропусками, хорошо исследованы \cite{OLSforGrouping}. 
Более сложный случай, когда вместо содержащих выбросы значений зависимой переменной наблюдаются номера классов(интервалов), в которые попадают эти наблюдения \cite{technometrics}, изучен не так хорошо и поэтому представляет больший интерес.
