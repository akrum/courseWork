\begin{center}
    \section*{ВВЕДЕНИЕ}
\end{center}
\phantomsection
\addcontentsline{toc}{section}{ВВЕДЕНИЕ}

В математической статистике широко используется регрессионная модель, т.е. такая модель, которая отражает зависимость некоторой величины от некоторых других независимых величин. 
Одним из примеров использования такой модели является автономный мониторинг целостности приемника (RAIM) системы глобального позиционирования (GPS). 
Это технология, которая разработана для оценки целостности сигналов GPS. 
Такая система очень важна в приложениях авиационной и морской навигации, где безопасность приложений GPS критична \cite{GPS_POSITIONING}.

В случае когда наблюдаемые данные полностью соответствуют модельным предположениям, в литературе известны несколько методов оценивания модельных параметров, такие как метод наименьших квадратов, метод максимального правдоподобия и др., которые дают состоятельные и несмещенные оценки. 
К сожалению, если наблюдаемые данные содержат искажения (т.е. данные, не удовлетворяющие модельным предположениям), то такие классические методы начинают давать смещенные и несостоятельные оценки, а иногда и вовсе неприменимы на практике. 
Поэтому актуальной является задача разработки новых оценок, учитывающих наличие конкретного вида искажения. 

В качестве искажений данных в литературе часто рассматриваются аномальные наблюдения, цензурирование, пропуски и др \cite{ComparisonRobust}. В рамках данной дипломной работы, будет рассмотрена модель регрессии с аномальными наблюдениями. 
Аномальными наблюдениями (выбросами) называются такие наблюдения, которые описываются распределением вероятностей, отличным от распределения вероятностей истинных значений.
Аномальные наблюдения могут возникать по разным причинам: из-за ошибки измерения, из-за необычной природы входных данных и др \cite{OLSforGrouping}.

В прошлом веке в работах Хьюбера была заложена теория робастного оценивания, т.~е. такого оценивания, которое исключает влияние искажений в выборке на результат оценивания.

В \cite{Huber} были предложены следующие робастные оценки:
\begin{itemize}
    \item М-Оценки;
    \item R-Оценки;
    \item L-Оценки.
\end{itemize}
М-оценки -- некоторое подобие оценок максимального правдоподобия (ММП-оценки - частный случай), L-оценки строятся на основе линейных комбинаций порядковых статистик, R-оценки -- на основе ранговых статистик.

Такие случаи, когда зависимые переменные наблюдаются с выбросами или с пропусками, хорошо исследованы \cite{OLSforGrouping}. 
Более сложный случай, когда вместо содержащих выбросы значений зависимой переменной наблюдаются номера классов(интервалов), в которые попадают эти наблюдения \cite{technometrics}, изучен не так хорошо и поэтому представляет больший интерес.

В дипломной работе мы будем рассматривать модель линейной регрессии с аномальными наблюдениями при наличии группирования выборки.