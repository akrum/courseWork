\begin{center}
    \section*{Заключение}
\end{center}
\phantomsection
\addcontentsline{toc}{section}{Заключение}
В ходе выполнения дипломной работы были получены следующие результаты: была рассмотрена математическая модель линейной регрессии с выбросами при наличии группирования наблюдений; были описаны основные методы оценивания параметров линейной регрессии при наличии выбросов: оценки МНК, М-оценки;
были построены оценки параметров линейной регрессии при наличии группирования наблюдений по методу максимального правдоподобия;
были проведены компьютерные эксперименты в которых построенные оценки применялись к модельным данным;
результаты экспериментов показали, что построенные оценки могут быть состоятельными.

В ходе дипломной работы был проведен аналитический обзор литературы методов статистического анализа данных при наличии классифицированных наблюдений с искажениями.
В результате был реализован альтернативный метод - \textit{метод наименьших квадратов по центрам интервалов}.

Был проведен сравнительный анализ альтернативного метода с оценками максимального правдоподобия. Оценки максимального правдоподобия с переклассификацией выборки показали наилучшие результаты. 

Над оценками максимального правдоподобия с переклассификацией выборки были осуществлены эксперименты, в которых изменялась константа $K$ для метода $K$-ближайших соседей (см.~п.~\ref{ss3_3_1}). Выяснилось, что увеличение константы $K$ повышает точность аппроксимации.

Реализованные методы максимального правдоподобия с переклассификацией и МНК по серединам интервалов  были обобщены на случай полиномиальной регрессии.

Был построен еще один способ переклассификации выборки, который показывает лучший результат по сравнению с методом K-ближайших соседей (по проведенным экспериментам оказалось, что метод в среднем правильно исправляет в 10 раз больше аномальных наблюдений).