\begin{center}
    \section*{Заключение}
\end{center}
\phantomsection
\addcontentsline{toc}{section}{Заключение}

В ходе преддипломной практики был проведен аналитический обзор литературы методов статистического анализа данных при наличии классифицированных наблюдений с искажениями.
В результате был реализован альтернативный метод - \textit{метод наименьших квадратов по центрам интервалов}.

Был проведен сравнительный анализ альтернативного метода с оценками максимального правдоподобия. Оценки максимального правдоподобия с переклассификацией выборки показали наилучшие результаты. 

Над оценками максимального правдоподобия с переклассификацией выборки были осуществлены эксперименты, в которых изменялась константа $K$ для метода $K-$ соседей (см.~п.~\ref{ss3_3_1}). Выяснилось, что увеличение константы $K$ повышает точность аппроксимации.

Реализованные методы максимального правдоподобия с переклассификацией и МНК по серединам интервалов  были обобщены на случай полиномиальной регрессии.


По проведенным экспериментам видно, что ОМП с переклассификацией показывают не хуже результаты, чем альтернативные оценки.
Можно добиться более точных результатов аппроксимации, если хорошо подобрать параметры оценок.