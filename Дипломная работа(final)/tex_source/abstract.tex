\begin{center}
    \section*{РЕФЕРАТ}
\end{center}
\phantomsection

Дипломная страница, ~.с, ~рис., 15 источников.

\textbf{Ключевые слова:} ЛИНЕЙНАЯ РЕГРЕССИЯ, АНОМАЛЬНЫЕ НАБЛЮДЕНИЯ, ГРУППИРОВАНЫЕ НАБЛЮДЕНИЯ, ОЦЕНКИ МАКСИМАЛЬНОГО ПРАВДОПОДОБИЯ.

\textbf{Объект исследование:} линейная регрессия с аномальными наблюдениями при наличии группированых наблюдений. Оценки ее параметров.

\textbf{Цель работы:} предложить способ оценивания параметров линейной регрессии с аномальными наблюдениями при наличии группированых наблюдений, устойчивый к аномальным наблюдениям.

\textbf{Основные методы исследования:} оценки максимального правдоподобия, метод секущих решения систем нелинейных уравнений, наименьший уровень выброса, случайный лес.

\newpage

\begin{center}
    \section*{Рэферат}
\end{center}
\phantomsection
Дыпломная работа

\newpage

\begin{center}
    \section*{ABSTRACT}
\end{center}
\phantomsection

Graduate work, ~.pages, ~figures., 15 sources.

\textbf{Key words:} LINEAR REGRESSION, OUTLIERS, SAMPLE CLUSTERING, MAXIMUM LIKELIHOOD ESTIMATES.

\textbf{Object of study:} linear regression with outliers in the presense of clustered observations . Estimates of its parameters.

\textbf{Objective:} propose a robust method of linear regression with outliers in the presense of clustered observations parameters estimation.

\textbf{Methods of research:} maximum likelihood estimates, secant method of solving nonlinear equations , least outlier factor, random forest.

\textbf{Result:}.

\textbf{The field of application:}.