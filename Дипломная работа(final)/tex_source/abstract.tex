\begin{center}
    \section*{РЕФЕРАТ}
\end{center}
\phantomsection

Дипломная страница, 37.с, 11 рис., 15 источников.

\textbf{Ключевые слова:} ЛИНЕЙНАЯ РЕГРЕССИЯ, АНОМАЛЬНЫЕ НАБЛЮДЕНИЯ, ГРУППИРОВАНЫЕ НАБЛЮДЕНИЯ, ОЦЕНКИ МАКСИМАЛЬНОГО ПРАВДОПОДОБИЯ.

\textbf{Объект исследование:} линейная регрессия с аномальными наблюдениями при наличии группированых наблюдений. Оценки ее параметров.

\textbf{Цель работы:} предложить способ оценивания параметров линейной регрессии с аномальными наблюдениями при наличии группированых наблюдений, устойчивый к аномальным наблюдениям.

\textbf{Основные методы исследования:} оценки максимального правдоподобия, метод секущих решения систем нелинейных уравнений, локальный уровень выброса, случайный лес.

\textbf{Результат:} Были предложены алгоритмы оценивания параметров линейной и полиномиальной регрессии с аномальными наблюдениями при наличии группирования выборки.

\newpage

\begin{otherlanguage}{belarusian}

\begin{center}
    \section*{Рэферат}
\end{center}
\phantomsection
Дыпломная праца, 37 cтаронак, 11 малюнкаў, 15 крыніц.

\textbf{Ключавыя словы:}  ЛІНЕЙНАЯ РЭГРЭСІЯ, АСТАНЦЫ, КЛАСТАРЫЗАЦЫЯ ВЫБАРКІ, АЦЭНКІ МАКСІМАЛЬНАГА ПРАЎДАПАДАБЕНСТВА.

\textbf{Аб'ект даследаванне:} лінейная рэгрэсія з анамальнымі назіраннямі пры наяўнасці групавання назіранняў. Ацэнкі яе параметраў.

\textbf{Мэта працы:} прапанаваць спосаб ацэньвання параметраў лінейнай рэгрэсіі з анамальнымі назіраннямі пры наяўнасці групавання назіранняў, ўстойлівы да анамальных назіранняў.

\textbf{Асноўныя метады даследавання:} ацэнкі максімальнага праўдападабенства, метад хорд рашэння сістэм нелінейных раўнанняў, лакальны ўзровень выкіду, выпадковы лес.

\textbf{Вынік:} Былі прапанаваны алгарытмы ацэньвання параметраў лінейнай і паліномнай рэгрэсіі з анамальнымі назіраннямі пры наяўнасці групавання выбаркі.

\end{otherlanguage}

\newpage

\begin{center}
    \section*{ABSTRACT}
\end{center}
\phantomsection

Graduate work, 37 pages, 11 figures., 15 sources.

\textbf{Key words:} LINEAR REGRESSION, OUTLIERS, SAMPLE CLUSTERING, MAXIMUM LIKELIHOOD ESTIMATES.

\textbf{Object of study:} linear regression with outliers in the presense of clustered observations . Estimates of its parameters.

\textbf{Objective:} propose a robust method of linear regression with outliers in the presense of clustered observations parameters estimation.

\textbf{Methods of research:} maximum likelihood estimates, secant method of solving nonlinear equations , local outlier factor, random forest.

\textbf{Result:} estimating algorithms of linear and polynomial regression with outliers in the presense of clustered observations parameters were designed.
