\begin{center}
    \section*{ВВЕДЕНИЕ}
\end{center}
\phantomsection
\addcontentsline{toc}{section}{ВВЕДЕНИЕ}

В математической статистике широко используется регрессионная модель.
Существует несколько подходов для оценки параметров регрессии, но далеко не все устойчивы к возникновениям аномальных наблюдений, 
то есть таких наблюдений, которые не подчиняются общей модели. 
В реальной жизни аномальные наблюдения возникают постоянно. 
Такие наблюдения могут возникать по разным причинам: из-за ошибки измерения, из-за необычной природы входных данных.
По этой причине большинство методов просто неприменимо.
В прошлом веке в работах Хьюбера была заложена теория робастного оценивания.

Такие случаи, когда зависимые переменные наблюдаются с выбросами или с пропусками, хорошо исследованы \cite{OLSforGrouping}. Более сложный случай, когда вместо содержащих выбросы значений зависимой переменной наблюдаются номера классов(интервалов), в которые попадают эти наблюдения \cite{technometrics}.
Темой курсового проекта было \textit{"Статистическое оценивание параметров линейной регрессии с выбросами при наличии группирования наблюдений"}.

Целью преддипломной практики было продолжение исследования и улучшение оценок, построенных в 
курсовом проекте. 
