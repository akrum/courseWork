\begin{center}
    \section*{ВВЕДЕНИЕ}
\end{center}
\phantomsection
\addcontentsline{toc}{section}{ВВЕДЕНИЕ}

Целью преддипломной практики было продолжение исследования и улучшение оценок, построенных в 
курсовом проекте. Темой курсового проекта было \textit{"Статистическое оценивание параметров линейной регрессии с выбросами при наличии группирования наблюдений"}.

Оценки были построенны с помощью максимизирования функции правдоподобия:
\begin{eqnarray}
    \label{eq22}l(\beta,\sigma^2, \nu_0,\dots, \nu_{k-1})&=&\textup{ln}(\prod_{i=1}^{n}P(\mu_i=j))=\\
    \label{eq23}&=&\sum_{i=1}^{n}\ln(P(\mu_i=j)),
\end{eqnarray}
где:
\begin{eqnarray}
    P(\mu_i=j)=P(y_i\in \nu_{\mu_i}),
\end{eqnarray}
$y_i$ - значения функции регрессии, а $\nu_0,\dots,\nu_{k-1}$ - номера полуинтервалов, разбивающих множество значений функции регрессии:
\begin{eqnarray}
    \label{eq1}(-\infty,a_1]\bigcup(a_1,a_2]\bigcup \dots \bigcup(a_{k-1},+\infty )=\mathcal{R}
\end{eqnarray}
$\mu_i$ номер полуинтервала, в который он попал $y_i$.
\begin{eqnarray}
    \mu_i=j, \textup{если $y_i$ отнесли к полуинтервалу $\nu_j$}.
\end{eqnarray}

Задача максимизирования решала с помощью решения нелинейной системы уравнений:
\begin{eqnarray}
    \frac{\delta l}{\delta \beta} = 0,
\end{eqnarray}
где
\begin{multline}
    \label{eq27}\frac{\delta l}{\delta \beta}=\frac{\delta \sum_{i=1}^{n}\ln(P(\mu_i=j))}{\delta \beta}=\frac{\delta \sum_{i=1}^{n}\ln P(y_i\in \nu_{\mu_i})}{\delta \beta}=~\\
    =\frac{\delta \sum_{i=1}^{n} \ln(\frac{1}{2}(\textup{erf}(\frac{a_{\mu_i+1}-f(x_i,\beta)}{\sqrt{2}\sigma})-\textup{erf}(\frac{a_{\mu_i}-f(x_i,\beta)}{\sqrt{2}\sigma})) )         }{\delta \beta}=~\\
    =  \sum_{i=1}^{n}\Big((1-(\delta_{\mu_i 0}+\delta_{\mu_i k-1}))\frac{(\textup{erf'}(\frac{a_{\mu_i+1}-f(x_i,\beta)}{\sqrt{2}\sigma})-\textup{erf'}(\frac{a_{\mu_i}-f(x_i,\beta)}{\sqrt{2}\sigma}))}{ (\textup{erf}(\frac{a_{\mu_i+1}-f(x_i,\beta)}{\sqrt{2}\sigma})-\textup{erf}(\frac{a_{\mu_i}-f(x_i,\beta)}{\sqrt{2}\sigma}))}+~\\
    +(\delta_{\mu_i 0}+\delta_{\mu_i k-1})\frac{\textup{erf'}(\frac{a_{\mu_i}-f(x_i,\beta)}{\sqrt{2}\sigma})}{(1+\textup{erf}(\frac{a_{\mu_i}-f(x_i,\beta)}{\sqrt{2}\sigma}))}\Big)  (-1) \frac{\delta f(x_i,\beta)}{\delta \beta} )=~
\end{multline}
\begin{multline}
    \nonumber 
    =-\sum_{i=1}^{n}\begin{pmatrix}
        1\\
        x_{i1}\\
        \dots\\
        x_{in}
    \end{pmatrix}\times  \Big((1-(\delta_{\mu_i 0}+\delta_{\mu_i k-1}))\frac{(\textup{erf'}(\frac{a_{\mu_i+1}-f(x_i,\beta)}{\sqrt{2}\sigma})-\textup{erf'}(\frac{a_{\mu_i}-f(x_i,\beta)}{\sqrt{2}\sigma}))}{ (\textup{erf}(\frac{a_{\mu_i+1}-f(x_i,\beta)}{\sqrt{2}\sigma})-\textup{erf}(\frac{a_{\mu_i}-f(x_i,\beta)}{\sqrt{2}\sigma}))}+~\\
    +(\delta_{\mu_i 0}+\delta_{\mu_i k-1})\frac{\textup{erf'}(\frac{a_{\mu_i}-f(x_i,\beta)}{\sqrt{2}\sigma})}{(1+\textup{erf}(\frac{a_{\mu_i}-f(x_i,\beta)}{\sqrt{2}\sigma}))}\Big).~
\end{multline}
