\begin{center}
    \section*{ВВЕДЕНИЕ}
\end{center}
\phantomsection
\addcontentsline{toc}{section}{ВВЕДЕНИЕ}

Целью преддипломной практики было продолжение исследования и улучшение оценок, построенных в 
курсовом проекте. Темой курсового проекта было \textit{"Статистическое оценивание параметров линейной регрессии с выбросами при наличии группирования наблюдений"}.

В ходе курсового проекта оценки строили для модели линейной регрессии с выбросами:
\begin{eqnarray}
    \label{eq3}y_i^{\widetilde{\varepsilon}}=(\xi_i)y_i+ (1-\xi_i)\eta_i,
\end{eqnarray}
где $\xi_i$ принимает значение, равное 1, с вероятностью $1-\widetilde{\varepsilon}$ и значение, равное 0, с вероятностью $\widetilde{\varepsilon}$, т.е.:
\begin{eqnarray}\label{eq4}
    \begin{cases}
        p(\xi_i=0)=\widetilde{\varepsilon},\\
        p(\xi_i=1)=1-\widetilde{\varepsilon},
    \end{cases},
\end{eqnarray}
$\eta_i$-случайная величина из некоторого вообще говоря неизвестного распределения.

$y_i^{\widetilde{\varepsilon}}$ -- $i$-е наблюдение из $N$ наблюдений($N$-объем выборки), $x_i=(x_{i1},x_{i2},\dots,x_{in})$ регрессоры, \{$\beta_k, k=\overline{0,n}$\}-- параметры регрессии, а $\varepsilon_i$ -- случайная ошибка $i$-го эксперемента, распределение которой подчиняется нормальному закону с нулевым математическим ожиданием и дисперсией $\sigma^2$.
\begin{eqnarray}
    \label{eq2}y_i= 
    \begin{pmatrix}
        \beta_0\\
        \beta_1\\
        \dots\\
        \beta_n
    \end{pmatrix}\times
    \begin{pmatrix}
        1\\
        x_{i1}\\
        \dots\\
        x_{in}
    \end{pmatrix}^{T}+ \varepsilon_i,
\end{eqnarray}

Параметр $\xi_i$ имеет следующий содержательный смысл: если $\xi_i=0$, то вместо истинного значения мы наблюдаем выброс, если $\xi_i=1$, то наблюдается истинное значение.
Переменную $\widetilde{\varepsilon}$ будем называть долей аномальных наблюдений. Величины $\xi_i, x_i$ и $\eta_i$ являются независимыми.

Каждый $y_i$ принадлежит нормальному распределению:
\begin{eqnarray}
    \label{eq12} y_i=f(x_i,\beta)+\varepsilon_i \sim \mathcal{N}(f(x_i,\beta),\sigma^2).
\end{eqnarray}

Разделим множество значений функции регрессии, т.е множество $\mathcal{R}$, на $k$ полуинтервалов:
\begin{eqnarray}
    \mathcal{R}=(-\infty,a_1]\bigcup(a_1,a_2]\bigcup \dots \bigcup(a_{k-1},+\infty ).
\end{eqnarray}
Обозначим полученные интервалы: $\nu_0,\dots,\nu_{k-1}$.

Далее в работе будем считать, что вместо истинных значений зависимых переменных $y_i$ наблюдается только номер класса, к которому это наблюдение попало.
Тогда для каждого $y_i$ будем наблюдать лишь номер полуинтервала $\mu_i$, в который он попал.
\begin{eqnarray}
    \label{eq13}\mu_i=j, \textup{если $y_i$ отнесли к полуинтервалу $\nu_j$}.
\end{eqnarray}

В курсовом проекте решалась задача статистического оценивания параметров модели \{$\beta_k, k=\overline{0,n}$\} по известным группированным наблюдениям с аномалиями.